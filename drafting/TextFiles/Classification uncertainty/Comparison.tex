\section{Comparison}

This section aims to compare and provide commentary on the presented approaches. Clearly, utilizing the standard deviation emerges as the more effective choice compared to the other methods. This superiority can be attributed to the higher precision and specificity of this metric. Additionally, the application of aleatoric uncertainty yields favorable outcomes, whereas epistemic uncertainty exhibits slightly inferior performance. The lower performance of the epistemic uncertainty was expected, given its reliance on the model rather than the input data.

In contrast, the uncertainty penalization approach proved to be ineffective, aligning with the notion that uncertainty is a metric about something unknown and, as such, cannot be used to acquire further information. Consequently, this approach will be abandoned.

Subsequent phases of this study will involve employing the classification strategies that implement the "I don't know" behavior. These methodologies will serve to assess the network robustness, which is the core objective of this research. A comprehensive evaluation of their efficacy will done.