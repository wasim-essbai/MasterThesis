The ASAP algorithm to automatically selects the points where to evaluate the accuracy. The best points to select would be those in which the accuracy curve intersects the threshold $\Theta$, to detect the bounds of the various intervals. However, the analytical form of the curve is not known a priori, and it is not possible to compute these intersections, so the idea is to select points as close as possible to $\Theta$.

The method is based on the assumption that, once computed the accuracy for two alteration levels A and B, the real curve between A and B is included in the area between two parabolas passing through the points A and B, and having concavity depth respectively $+\hat{a}$ and $-\hat{a}$. If there is an intersection between that area and the threshold, and the distance between A and B is sufficiently large, then it's possible to compute the accuracy in the middle point M between A and B, and add it to the sample set. Now it's possible to apply the same procedure recursively on the intervals $[A, M]$ and $[M, B]$. In this way, the number of evaluated points is adaptively determined and depends on the value of the parameter $\hat{a}$. Intuitively, the higher is the value of $\hat{a}$ the higher is the number of alteration levels evaluated by the algorithm, that means that it must choosed based on the robustness estimation accuracy needed and available resources.

At the end of the algorithm execution there is a set $RES = \{ \langle l_1, acc_1 \rangle, ...,\langle l_n, acc_n \rangle\}$, starting which it's possible to compute the robustness generalizing Def. 3 as follows:
\[
	rob_A(C,P) = \frac{\sum_{j = 2}^{n} H(acc_j - \Theta)(l_j - l_{j-1})}{U_A - L_A}
\]

It's possible to show that ASAP uses fewer alteration levels than the uniform sampling, so it saves time, but it still provides an accurate approximation of the robustness.

\subsection{Maximum Error Estimation of the Computed Robustness}

With the presented method it's possible to quantify the maximum error in the robustness estimation. Let be $IP = \{(l_j,l_{j+1}) | \langle l_j, acc_j \rangle, \langle l_{j+1}, acc_{j+1} \rangle  \in RES \land (parabIntsct(l_j, acc_j,l_{j+1}, acc_{j+1}, \hat{a}, \Theta) \lor parabIntsct(l_j, acc_j,l_{j+1}, acc_{j+1}, -\hat{a}, \Theta))\}$ the pairs set of two consecutive points from RES such that the parabolas passing from them with concavity depth $\pm \hat{a}$ intersect the threshold $\Theta$. Intuitevely, this means that the real curve may intersect, but ASAP has quit sampling because the two points have alteration levels sufficiently close ($l_{j+1} - l_j < minStep$). Therefore, assuming a correct value for $\hat{a}$, the error that could be done comes only from those intervals in IP. Formally

\textbf{Theorem 1.} Let C be a CNN and A a given alteration type defined in the range $[L_A, U_A]$. Let $rob_A$ be the robustness computed for C and A by ASAP using a given $\hat{a}$. Let $rob_A^*$ be the real robustness value. Under the assumption that $\hat{a}$ is a suitable parameter, i.e., the real accuracy curve is included in the areas of two parabolas with concavity depth $\hat{a}$, the maximum error of the computed robustness has a guaranteed upper bound defined as follows:
\[
	|rob_A - rob_A^*| \leq \epsilon_A
	\ \ \ with \ \
	\epsilon_A = \frac{\sum_{(l_j, l_{j+1}) \in IP} (l_{j+1} - l_j)}{|U_A - L_A|}
\]